\documentclass[journal,12pt,twocolumn]{IEEEtran}
%
\usepackage{setspace}
\usepackage{gensymb}
%\doublespacing
\singlespacing

%\usepackage{graphicx}
%\usepackage{amssymb}
%\usepackage{relsize}
\usepackage[cmex10]{amsmath}
%\usepackage{amsthm}
%\interdisplaylinepenalty=2500
%\savesymbol{iint}
%\usepackage{txfonts}
%\restoresymbol{TXF}{iint}
%\usepackage{wasysym}
\usepackage{amsthm}
\usepackage{mathrsfs}
\usepackage{txfonts}
\usepackage{stfloats}
\usepackage{steinmetz}
\usepackage{bm}
\usepackage{cite}
\usepackage{cases}
\usepackage{subfig}
%\usepackage{xtab}
\usepackage{longtable}
\usepackage{multirow}
%\usepackage{algorithm}
%\usepackage{algpseudocode}
\usepackage{enumitem}
\usepackage{mathtools}
\usepackage{tikz}
\usepackage{circuitikz}
\usepackage{verbatim}
\usepackage{tfrupee}
\usepackage[breaklinks=true]{hyperref}
%\usepackage{stmaryrd}
\usepackage{tkz-euclide} % loads  TikZ and tkz-base
%\usetkzobj{all}
\usepackage{listings}
    \usepackage{color}                                            %%
    \usepackage{array}                                            %%
    \usepackage{longtable}                                        %%
    \usepackage{calc}                                             %%
    \usepackage{multirow}                                         %%
    \usepackage{hhline}                                           %%
    \usepackage{ifthen}                                           %%
  %optionally (for landscape tables embedded in another document): %%
    \usepackage{lscape}     
\usepackage{multicol}
\usepackage{chngcntr}
%\usepackage{enumerate}

%\usepackage{wasysym}
%\newcounter{MYtempeqncnt}
\DeclareMathOperator*{\Res}{Res}
%\renewcommand{\baselinestretch}{2}
\renewcommand\thesection{\arabic{section}}
\renewcommand\thesubsection{\thesection.\arabic{subsection}}
\renewcommand\thesubsubsection{\thesubsection.\arabic{subsubsection}}

\renewcommand\thesectiondis{\arabic{section}}
\renewcommand\thesubsectiondis{\thesectiondis.\arabic{subsection}}
\renewcommand\thesubsubsectiondis{\thesubsectiondis.\arabic{subsubsection}}

% correct bad hyphenation here
\hyphenation{op-tical net-works semi-conduc-tor}
\def\inputGnumericTable{}                                 %%

\lstset{
%language=C,
frame=single, 
breaklines=true,
columns=fullflexible
}
%\lstset{
%language=tex,
%frame=single, 
%breaklines=true
%}

\begin{document}
%


\newtheorem{theorem}{Theorem}[section]
\newtheorem{problem}{Problem}
\newtheorem{proposition}{Proposition}[section]
\newtheorem{lemma}{Lemma}[section]
\newtheorem{corollary}[theorem]{Corollary}
\newtheorem{example}{Example}[section]
\newtheorem{definition}[problem]{Definition}
%\newtheorem{thm}{Theorem}[section] 
%\newtheorem{defn}[thm]{Definition}
%\newtheorem{algorithm}{Algorithm}[section]
%\newtheorem{cor}{Corollary}
\newcommand{\BEQA}{\begin{eqnarray}}
\newcommand{\EEQA}{\end{eqnarray}}
\newcommand{\define}{\stackrel{\triangle}{=}}
\bibliographystyle{IEEEtran}
%\bibliographystyle{ieeetr}
\providecommand{\mbf}{\mathbf}
\providecommand{\pr}[1]{\ensuremath{\Pr\left(#1\right)}}
\providecommand{\qfunc}[1]{\ensuremath{Q\left(#1\right)}}
\providecommand{\sbrak}[1]{\ensuremath{{}\left[#1\right]}}
\providecommand{\lsbrak}[1]{\ensuremath{{}\left[#1\right.}}
\providecommand{\rsbrak}[1]{\ensuremath{{}\left.#1\right]}}
\providecommand{\brak}[1]{\ensuremath{\left(#1\right)}}
\providecommand{\lbrak}[1]{\ensuremath{\left(#1\right.}}
\providecommand{\rbrak}[1]{\ensuremath{\left.#1\right)}}
\providecommand{\cbrak}[1]{\ensuremath{\left\{#1\right\}}}
\providecommand{\lcbrak}[1]{\ensuremath{\left\{#1\right.}}
\providecommand{\rcbrak}[1]{\ensuremath{\left.#1\right\}}}
\theoremstyle{remark}
\newtheorem{rem}{Remark}
\newcommand{\sgn}{\mathop{\mathrm{sgn}}}
\providecommand{\abs}[1]{\left\vert#1\right\vert}
\providecommand{\res}[1]{\Res\displaylimits_{#1}} 
\providecommand{\norm}[1]{\left\lVert#1\right\rVert}
%\providecommand{\norm}[1]{\lVert#1\rVert}
\providecommand{\mtx}[1]{\mathbf{#1}}
\providecommand{\mean}[1]{E\left[ #1 \right]}
\providecommand{\fourier}{\overset{\mathcal{F}}{ \rightleftharpoons}}
%\providecommand{\hilbert}{\overset{\mathcal{H}}{ \rightleftharpoons}}
\providecommand{\system}{\overset{\mathcal{H}}{ \longleftrightarrow}}
	%\newcommand{\solution}[2]{\textbf{Solution:}{#1}}
\newcommand{\solution}{\noindent \textbf{Solution: }}
\newcommand{\cosec}{\,\text{cosec}\,}
\providecommand{\dec}[2]{\ensuremath{\overset{#1}{\underset{#2}{\gtrless}}}}
\newcommand{\myvec}[1]{\ensuremath{\begin{pmatrix}#1\end{pmatrix}}}
\newcommand{\mydet}[1]{\ensuremath{\begin{vmatrix}#1\end{vmatrix}}}
%\numberwithin{equation}{section}
\numberwithin{equation}{subsection}
%\numberwithin{problem}{section}
%\numberwithin{definition}{section}
\makeatletter
\@addtoreset{figure}{problem}
\makeatother
\let\StandardTheFigure\thefigure
\let\vec\mathbf
%\renewcommand{\thefigure}{\theproblem.\arabic{figure}}
\renewcommand{\thefigure}{\theproblem}
%\setlist[enumerate,1]{before=\renewcommand\theequation{\theenumi.\arabic{equation}}
%\counterwithin{equation}{enumi}
%\renewcommand{\theequation}{\arabic{subsection}.\arabic{equation}}
\def\putbox#1#2#3{\makebox[0in][l]{\makebox[#1][l]{}\raisebox{\baselineskip}[0in][0in]{\raisebox{#2}[0in][0in]{#3}}}}
     \def\rightbox#1{\makebox[0in][r]{#1}}
     \def\centbox#1{\makebox[0in]{#1}}
     \def\topbox#1{\raisebox{-\baselineskip}[0in][0in]{#1}}
     \def\midbox#1{\raisebox{-0.5\baselineskip}[0in][0in]{#1}}
\vspace{3cm}
\title{
ASSIGNMENT 1 - EE5600
	}
\author{ RS Girish - EE20RESCH14005$^{*}$% <-this % stops a space
	}	

\maketitle
\newpage
\tableofcontents
\bigskip
\renewcommand{\thefigure}{\theenumi}
\renewcommand{\thetable}{\theenumi}

\begin{abstract}
This paper contains solution to problem no 17 of Lines and Planes section.
Links to Python codes are available below.  
\end{abstract}
Download python codes using 
\begin{lstlisting}
https://github.com/rsgirishkumar/Assignment1/codes/
\end{lstlisting}
\section{Problem}
Find $m$ if 
\begin{align}
\myvec{2 & 3 }\vec{x}&=11\\
\myvec{2 & -4 }\vec{x}&=-24\\
\myvec{m & -1 }\vec{x}&=-3
\end{align}
\section{Solution}
Given, the system of equations in matrix equation format are as below
\begin{align}
\myvec{2 & 3\\2 & -4\\m & -1}
\vec{x}=
\myvec{11\\-24\\-3}
\end{align}
\textbf{Step1}: Assuming the system of equations are consistent,Since there is an unknown $m$ in equation 2.0.1, $m$ is to found first. 
\\To Find $m$, find $x$ and $y$ using ratio of determinants methods by forming a 2x2 matrix as below, i.e. A$x$=B format.\\
\begin{align}
\myvec{2 & 3\\2 & -4}
\myvec{x\\y}=
\myvec{11\\-24}
\end{align}
\\Since the system of equations are assumed consistent, the x and y values should satisfy the 1.0.3 equation i.e.
\begin{align*}
\myvec{m & -1}\vec{x}&=-3\\
\end{align*}
As per ratio of determinants,\\
\begin{align*}
\vec{x}=\frac{\begin{vmatrix}11 & 3\\-24 & -4\\\end{vmatrix}}{\begin{vmatrix}2 & 3\\2 & -4\\\end{vmatrix}}=\frac{-44+72}{-8-6}=\frac{28}{-14}=-2\\
\vec{y}=\frac{\begin{vmatrix}2 & 11\\2 & -24\\\end{vmatrix}}{\begin{vmatrix}2 & 3\\2 & -4\\\end{vmatrix}}=\frac{-48-22}{-8-6}=\frac{70}{14}=5\\
\end{align*}
The solution is\\
\begin{align}
\myvec{x\\y}\vec{=}\myvec{-2\\5}
\end{align}
On back-substituting the values of x and y in 1.0.3 equation i.e.\\
\begin{align}
\myvec{m & -1 }\vec{x}&=-3
\end{align}
The equation can be re-written as\\
\begin{align*}
\begin{pmatrix}m & -1\\\end{pmatrix} \begin{pmatrix}-2\\5\\\end{pmatrix}=-3
\end{align*}
\begin{align}
\Rightarrow m=-1
\end{align}
The third line equation can be substituted with intersection point for verification of solution.
\begin{align*}
\myvec{-1 & -1}&*\myvec{-2\\5}=-3
\end{align*}
Since, the intersection point satisfies the equation, it is one of the solution of all the three equations i.e.
\begin{align*}
\myvec{x\\y}=\myvec{-2\\5}
\end{align*}
So the system of equations can be re-written as\\
\begin{align}
\myvec{2 & 3 }\vec{x}&=11\\
\myvec{2 & -4 }\vec{x}&=-24\\
\myvec{-1 & -1 }\vec{x}&=-3
\end{align}
\\
and their matrix equation format is
\begin{align}
\myvec{2 & 3\\2 & -4\\-1 & -1}
\vec{x}=
\myvec{11\\-24\\-3}
\end{align}
\textbf{Step2}:Check the consistency of equations by using the rank of augumented matrix,M=[A'b] and N=matrix [A] as below.
\begin{align*}
N=\myvec{2 & 3\\2 & -4\\-1 & -1}\\
M=\myvec{2 & 3 & 11\\2 & -4 & -24\\-1 & -1 & -3}
\end{align*}
Calculating the rank of matrix M:
\begin{align*}
\myvec{2 & 3 & 11\\2 & -4 & -24\\-1 & -1 & -3}
\\R2 -> R2-R1\\
\myvec{2 & 3 & 11\\0 & -7 & -35\\-1 & -1 & -3}
\\R3 -> 2R3+R1\\
\myvec{2 & 3 & 11\\0 & -7 & -35\\0 & 1 & 5}
\\R3 -> R2+7R3\\
\myvec{2 & 3 & 11\\0 & -7 & -35\\0 & 0 & 0}
\end{align*}
\\No of non zero rows = 2.
\\Hence Rank of matrix(M) =2.
\\Calculating the rank of matrix N:
\begin{align*}
\myvec{2 & 3\\2 & -4\\-1 & -1}
\\R2 -> R2-R1\\
\myvec{2 & 3\\0 & -7\\-1 & -1}
\\R3 -> 2R3+R1\\
\myvec{2 & 3\\0 & -7\\0 & 1}
\\R3 -> R2+7R3\\
\myvec{2 & 3\\0 & -7\\0 & 0}
\end{align*}
\\No of non zero rows = 2.
\\Hence Rank of matrix(N) =2.\\
\\
Since rank of matrix N =2 and M =2, the system of equations are consistent.But,rank(N) = 2 is not equal to total no of rows in matrix(N) i.e.m =3.Since rank(N) != m there exist infinite number of solutions. Of which, the intersection point of any two equations is one of the solutions. From equation 2.0.9, the intersection point of lines
\begin{align*}
\myvec{2 & 3\\2 & -4\\-1 & -1}
\myvec{x\\y}=
\myvec{11\\-24\\-3}
\end{align*}
\\is equation 2.0.3 i.e.
\begin{align*}
\myvec{x\\y}=\myvec{-2\\5}
\end{align*}
The same can be verified from the plot of vectors as below.\\
\\
\textbf{Step3}:The vectors of equations are plotted on 2D axes by taking intersecting points on x and y axes respectively.Intersecting point is given in code.\\
\begin{lstlisting}
https://github.com/rsgirishkumar/Assignment1/codes/assignment1_solution.py
\end{lstlisting}
\begin{figure}[htbp]
  \centering \includegraphics[width=\columnwidth]{assignment1solution_graph1.png}\label{fig0}
  \caption{Three lines intersecting at a point.}\label{cap2}
  \label{fig:Intersection point (-2,5)}
\end{figure} 
\begin{figure}[htbp]
  \centering \includegraphics[width=\columnwidth]{assignment1solution_graph.png}\label{fig1}
  \caption{A Clear view.}\label{cap1}
  \label{fig:Intersection point (-2,5)}
\end{figure} 
\end{document}
